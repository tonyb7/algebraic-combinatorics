% Friday 1.19
\section{Expander Graphs}
Vaguely speaking, an expander graph is a sparse graph (\textit{i.e.}, it has few edges) which is nonetheless highly connected.
Expander graphs have applications to
\begin{itemize}
\item constructing error-correcting codes
\item derandomize algorithms
\item serve methods in number theory
\item hyperbolic manifolds
\end{itemize}

\begin{definition}[edge expansion ratio]
Lety \(G\) be a graph.
For \(S\subseteq V(G)\) denote the set of edges with one endpoint in \(S\) and one endpoint not in \(S\) by \(\partial S\).
Define the edge expansion ratio
\[ h(G)\coloneqq \min_{\substack{S\subseteq V(G)\\|S|\leq|V(G)|/2}} \frac{|\partial S|}{|S|} \]
\end{definition}
\begin{definition}
A sequence of \(k\)-regular graphs \((G_n)_{n\in\N}\) is called a family of expander graphs if
\[ |V(G_0)|<|V(G_1)|<\cdots \]
and there exists \(\varepsilon>0\) such that \(h(G_n)\geq0,\forall n\in\N\).
\end{definition}
\begin{remark}
It's relatively easy to show that such families exist by probabilistic arguments, but it's relatively difficult to construct them explicitly.
\end{remark}
\begin{theorem}
Suppose \(G\) is connected and \(k\)-regular so \(\lambda_1(G)=k\), then
\[ \frac{k-\lambda_2(G)}{2}\leq h(G)\leq \sqrt{2k(k-\lambda_2(G))} \]
\end{theorem}
Therefore a family \((G_n)_{n\in\N}\) of \(k\)-regular graphs with \(|V(G_0)|<|V(G_1)|<\cdots\) forms a family of expander graphs \iff there exists \(\varepsilon'>0\) such that
\[ k=\lambda_2(G_n)\geq\varepsilon' \quad \forall n\in\N \]
The spectral gap \(k-\lambda_2(G)\) or sometimes \( \min\{|k-\lambda_2{G}|,k-|\lambda_n(G)|\} \) measures how rapidly a random walk on \(G\) mixes.



\chapter{Tilings, Spanning Trees and Electrical Networks}
\begin{itemize}
\item motivation (and some results) come from Physics, Chemistry and Computer Science.
\end{itemize}

\section{Tilings and Perfect Matchings}
We'll prove the following theorems.
\begin{theorem}[Kasteleyn (1961)]
Let \(m,n\in\N\) be such that \(mn\) is even.
The number of domino tilings of an \(m\times n\) board equals
\[ T(m,n)=\prod_{j=1}^{m}\prod_{k=1}^n \left(4\cos^2\left(\frac{\pi j}{m+1}\right) + 4\cos^2\left(\frac{\pi k}{n+1}\right)\right)^{\frac{1}{4}} \]
\end{theorem}

\begin{definition}[Perfect matching]
A \emph{perfect matching} of a graph \(G\) is a subset of its edges which meets every vertex exactly once.
\end{definition}
\begin{remark}
Only graphs with even number of vertices can have a perfect matching.
\end{remark}