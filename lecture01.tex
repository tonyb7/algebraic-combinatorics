\begin{itemize}
\item Linear algebra
\item Group theory (Cayley graphs, Dynkin diagrams)
\end{itemize}

\section{Eigenvalues}

\begin{definition}[Adjacency matrix]
Let \(G=(V,E)\) be a finite graph.
The \emph{adjacency matrix} \( A(G)=\{0,1\}^{V\times V} \) is defined by
\[ A(G)_{v,w}=\begin{cases} 1 & \text{if } v\sim w \\ 0 & \text{otherwise} \end{cases} \]
\end{definition}

Recall the characteristic polynomial of a square matrix \(M\) over \C
\[ \phi_M(t)\coloneqq \det(tI-M). \]


\begin{definition}[Characteristic polynomial]
Let \(G=(V,E)\) be a finite graph.
The \emph{Characteristic polynomial} is defined by
\[ \phi_G(t)\coloneqq \phi_{A(G)}(t), \]
and call the zeros (with multiplicitier) the eigenvalues of \(G\).
\end{definition}

\begin{definition}[Spectrum]
The \emph{spectrum} of \(G\) is the multiset of its eigenvalues.
\end{definition}

\begin{example}[spectrum]
\begin{align*}
A(G)&=\begin{bmatrix} 0 & 1 \\ 1 & 0 \end{bmatrix}\\
\phi_G(t)&=\det\begin{pmatrix} t & -1 \\ -1 & t \end{pmatrix}=-t^2-1\\
\text{spectrum}&: \{1,-1\}
\end{align*}
\end{example}

\begin{remark}
Eigenvalues provide information about the connectivity of the graph.
\end{remark}


\begin{theorem}
Let \(A\in\R^{n\times n}\) be symmetric, then \(A\) \emph{orthogonally diagonalizable} over \R, \text{i.e.},
there exists an orthonormal basis
\[ v^{(1)}, v^{(2)},\cdots,v^{(n)}\in\R^n \]
of eigenvectors of \(A\) corresponding to real eigenvalues \(\lambda_1,\lambda_2,\cdots,\lambda_n\in\R\), we have
\[A=\sum_{i=1}^{n}\lambda_i v^{(i)}v^{(i)T} \]
\end{theorem}

Therefore the eigenvalues of any graph \(G\) are all real and we'll denote then
\[ \lambda_1(G)\geq\lambda_2(G)\geq\cdots\geq\lambda_n(G), \]
where \(n=|V(G)|\).

\begin{theorem}[Perron-Forbenius]
If a matrix \(A\in\R^{n\times n}\) has nonnegative entries, then the spectral radius of \(A\) (\textit{i.e.}, the maximum magnitude over all complex eigenvalues of A) is an eigenvalue of \(A\), corresponding to an eigenvector in \(\R_{\ge0}^n\).
\end{theorem}
Therefore for any graph \(G\), \(\lambda_1(G)\) is the spectral radius and corresponds to an eigenvector with nonnegative entries.
Perron-Forbenius also implies if \(G\) is connected, then \(\lambda_1(G)\) has multiplicity 1.


\begin{definition}[disjoint union]
If \(G=(V,E)\), \(G'=(V',E')\) are graphs, their \emph{disjoint union} is the graph
\[ G\sqcup G'=(V\sqcup V', E\sqcup E') \]
and
\[ A(G\sqcup G)=\begin{bmatrix} A(G) & 0 \\ 0 & A(G') \end{bmatrix} \]
\end{definition}
\begin{remark}
Spectrum doesn't detect if the graph is connected.
\end{remark}
\begin{example}
\begin{align*}
G&: V=\{1,2\}, E=\{\{1,2\}\}\\
\text{spectrum}&: \{1,-1\}\\
G\sqcup G&: V=\{1,2,3,4\}, E=\{\{1,2\},\{3,4\}\}\\
\text{spectrum}&: \{1,1,-1,-1\}
\end{align*}
\end{example}

\begin{remark}
Note that the spectrum of \(G\sqcup G'\) is the multiset union of the spectra of \(G\) and \(G'\).
\end{remark}


\section{Regular Graphs}
\begin{definition}
A graph \(G=(V,E)\) is called \emph{\(k\)-regular} if every vertex has degree \(k\in\N\).
\end{definition}
\begin{remark}
\(G\) is regular \iff \(\bar e\) is an eigenvector, where \(\bar e=\begin{bmatrix} 1 & 1 & \cdots & 1 \end{bmatrix}^T\in\R^n \).
\[ A(G)\cdot\bar e=\deg_G(V) \]
\end{remark}

\begin{proposition}
Let \(G=(V,E)\), then
\[ \E_{v\in V(G)} \deg_G(v) \stackrel{(i)}{\leq} \lambda_1(G) \stackrel{(ii)}{\leq} \max_{v\in V(G)}\deg_G(V), \]
where
\[ \E_{v\in V(G)} \deg_G(v)=\frac{1}{n}\sum_{v\in V(G)}\deg_G(v). \]
We have equality in (i) \iff \(G\) is regular.
If \(G\) is connected, equality holds in (ii) \iff \(G\) is regular.
\end{proposition}
\begin{proof}
\begin{enumerate}[(i)]
\item For any symmetric \(A\in\R^{n\times n}\),
\[ \lambda_{\max}(A)\geq x^T Ax, \forall x\in\R^n \text{ with } ||x||=1 \]
with equality \iff \(x\) is an eigenvector of \(A\) with eigenvalue \(\lambda_{\max}(A)\).
\begin{align*}
A&=\sum_j \lambda_j x^{(j)}x^{(j)T}\\
x&=\sum_j c_j x^{(j)}\\
\sum_j c_j^2&=1\\
x^T Ax&=\sum_j \lambda_j c_j^2
\end{align*}
Take \(A=A(G), x=\frac{1}{\sqrt{n}}\bar e\), then
\[ x^T Ax=\frac{1}{n} \underbrace{\bar e^T A\bar e}_{\text{sum of entries}}=\text{average degree} \]
\item Let's assume \(V(G)=[n]=\{1,2,\cdots,n\}\).
Recall that by Perron-Forbenius, \(\lambda_1(G)\) corresponds to an eigenvector \(x\in\R_{\ge0}^n\).
Then for any \(i\in[n]\),
\[ \lambda_1(G) x_i=\Big(A(G)x\Big)_i=\sum_{j=1}^{n} A(G)_{i,j}x_j=\sum_{j\sim i}x_j. \]
Now take \(i\in[n]\) so that \(x_i\) is maximum.
Then
\[ \lambda_1(G)x_i=\sum_{j\sim i}x_j\leq \sum_{j\sim i}x_i=\deg_G(v_i) x_i\leq \left( \max_{v\in V(G)}\deg_G(v) \right) x_i \tag{*} \]
So
\[ \lambda_1(G)\leq\max_{v\in V(G)}\deg_G(v). \]
Suppose equality holds everywhere in (ii) and \(G\) is connected, then equality holds everywhere in (*).
So \(x_j=x_i\) for all \(j\sim i\).
Applying the same argument to \(j\sim i\), then \(x_h=x_j=x_i\) for any \(h\sim j\), \texttt{etc}.
Since \(G\) is connected, \(x\) is a multiple of \(\bar e\), \textit{i.e.}, \(\bar e\) is an eigenvector of \(G\).
\end{enumerate}
\end{proof}
\begin{corollary}
The complete graph \(K_n\) is the only graph on \(n\) vertices with eigenvalue \(n-1\).
In particular, \(K_n\) is uniquely determined by its spectrum.
\end{corollary}
\begin{remark}
Not all graphs can be recovered by their spectrum, \textit{i.e.}, the spectrum is not a faithful (in that graph can be uniquely determined) graph invariant (in that spectrum depends on graph only up to isomorphism).
\end{remark}