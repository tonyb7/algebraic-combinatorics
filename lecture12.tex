
\begin{theorem}
$$N(a,b,c) = \prod_{i=1}^a\prod_{j=1}^b\prod_{k=1}^c \dfrac{i+j+k-1}{i+j+k-2}$$
\end{theorem}
\begin{proof}
We will apply Jacobi's Identity:
$$\det(A)\det(A_{[2,n-1][2,n-1]})=\det(A_{[2,n][2,n]})\det(A_{[1,n-1][1,n-1]}) - \det(A_{[2,n][1,n-1]})\det(A_{[1,n-1][2,n]})$$
for any $n\times n$ matrix $A$. This is a special case of general determinant identities called Grassmann-pliicker relations (or Schouter(???) identities in physics). In our case, we have
$$N(a,b,c)N(a,b,c-2) = N(a,b,c-1)N(a,b,c-1)-N(a-1,b+1,c-1)N(a+1,b-1,c-1),$$
where $N(a,b,0)=1, N(a,b,c) = {a+b\choose a}$. 
Now just need to check that
$$\prod_{i=1}^a\prod_{j=1}^b\prod_{k=1}^c \dfrac{i+j+k-1}{i+j+k-2}$$
satisfies the same recursion.
\end{proof}
There is a direct way to relate determinants to enumerating perfect matchings.

\section{EDIT: Aztec Diamond}
The $n^{\textrm{th}}$ Aztec diamond is a symmetric board with rows of length $2,4,6,\ldots,2n,2n,\ldots,6,4,2$.
\begin{theorem}
The $n^\textrm{th}$ Aztec diamond has $2^{n+1\choose 2}$ domino tilings.
\end{theorem}

\section{Spanning Trees}
\begin{theorem}
Let $m,n\in\mathbb{N}$ be odd. The number of domino tilings of an $m\times n$ board with one corner square removed is 
$$\prod_{j=1}^{\floor{\frac{m}{2}}}\prod_{k=1}^{\floor{\frac{n}{2}}}(4\cos^2(\frac{\pi j}{m+1})+4\cos^2(\frac{\pi k}{n+1}))$$
\end{theorem}
To prove this theorem we first construct a bijection from domino tilings to spanning trees of a grid graph. Then we will count spanning trees using the matrix-tree theorem.
\begin{definition}
A spanning subgraph $H$ of $G$ is a graph $V(H)=V(G)$ and $E(H)\subset E(G)$. If $H$ is a tree, we call $H$ a spanning tree of $G$.
\end{definition}
\begin{theorem}
let $m,n\in\mathbb{Z}_{>0}$, define a map $\varphi$ from the set of domino tilings of a $(2m-1)(2n-1)$ board whose north east corner is removed to the set of spanning trees of $P_m\square P_n$ as follows: 

Embed $P_m\square P_n$ inside the (2m-1)(2n-1) board. Given a domino tiling $M$, let $\varphi(M)$ be the spanning tree whose edges cross only one edge of $M$. This $\varphi$ is a bijection.
\end{theorem}
\begin{proof}
First we prove $\varphi$ is well-defined. This can be seen from the fact that 
\begin{enumerate}
\item the edges of $\varphi(M)$ correspond to  the vertices of $P_m\square P_n$ except for the northeast corner vertex. So $\varphi(M)$ has $mn-1$ edges,
\item $\varphi(M)$ is acyclic, otherwise, the path would enclose odd number of squares, which cannot be filled with dominoes.
\end{enumerate}

Then we must construct the unique domino tiling $M$ with $\varphi(M)=T$.
Root $T$ at the northeast corner vertex and direct all edges toward the root. This allows us to fill in the dominoes along the tree. 

Now we need to fill in the remainder. The remaining squares can be divided into several connected component $H_1,\ldots, H_l$ ( of the dual graph of the board). Note that each $H_i$ is acyclic, otherwise $H_i$ would divide $T$ into $\geq 2$ connected components. So, the all $H_i$ are trees.

Similarly, each $H_i$ intersects the boundary of the boundary of the board exactly once. Therefore each $H_i$ has a unique root. By the same method we could tile all the area.
\end{proof}