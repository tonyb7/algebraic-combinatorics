\begin{theorem}
$$N(a,b,c) = \Pi_{i=1}^a\Pi_{j=1}^b\Pi_{k=1}^c \dfrac{i+j+k-1}{i+j+k-2}$$
\end{theorem}
\begin{proof}
We will apply Jacobi's Identity:
$$\det(A)\det(A_{[2,n-1][2,n-1]})=\det(A_{[2,n][2,n]})\det(A_{[1,n-1][1,n-1]}) - \det(A_{[2,n][1,n-1]})\det(A_{[1,n-1][2,n]})$$
for any $n\times n$ matrix $A$. This is a special case of general determinant identities called Grassmann-pliicker relations (or Schouter(???) identities in physics)
$$N(a,b,c)N(a,b,c-2) = N(a,b,c-1)N(a,b,c-1)-N(a-1,b+1,c-1)N(a+1,b-1,c-1)$$,
where $N(a,b,0)=1, N(a,b,c) = ()$
\end{proof}

\section{Spanning Trees}
\begin{theorem}
Let $m,n\in\mathbb{N}$ be odd. The number of domino tilings of an $m\times n$ board with one corner square removed is 
$$\prod_{j=1}^{\floor{\frac{m}{2}}}\prod_{k=1}^{\floor{\frac{n}{2}}}(4\cos^2(\frac{\pi j}{m+1})+4\cos^2)(\frac{\pi k}{n+1})$$
\end{theorem}
To prove this theorem we first construct a bijection from domino tilings to spanning trees of a grid graph. Then we will count spanning trees using the matrix-tree theorem.
\begin{definition}
A spanning subgraph $H$ of $G$ is a graph $V(H)=V(G)$ and $E(H)\subset E(G)$. If $H$ is a tree, we call $H$ a spanning tree of $G$.
\end{definition}
\begin{theorem}
let $m,n\in\mathbb{Z}_{>0}$, define a map $\varphi$ from the set of domino tilings of a $(2m-1)(2n-1)$ board whose north east corner is removed to the set of spanning trees of $P_m\square P_n$ as follows: 

Embed $P_m\square P_n$ inside the (2m-1)(2n-1) board. Given a domino tiling $M$, let $\varphi(M)$ be the spanning three whose edges cross only one edge of $M$. This $\varphi$ is a bijection.
\end{theorem}
\begin{proof}
First we show that $\phi$ is well defined. Note that the edges of $\varphi(M)$ correspond to the vertices of $P_m\square P_n$ except for the northeast coner vertex. So $\varphi(m)$ has $mn-1$ edges.
\end{proof}