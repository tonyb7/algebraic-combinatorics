\section{Graph Homomorphisms}
\begin{definition}
Let $G$ and $H$ be graphs. We say that $\phi:V(G)\to V(H)$ is a homomorphism if
\[ \forall v,w\in V(G), \quad v\sim_{G} w\implies \phi(v)\sim_{H} \phi(w). \]
We call $\phi$ an isomorphism if $\phi$ is a bijection and its inverse $\phi^{-1}$ is also a homomorphism.
If \(\phi\) is a bijection, it is an isomorphism \iff
\[ \forall v,w\in V(G),\quad v\sim_{G} w \Longleftrightarrow \phi(v)\sim_{H} \phi(w) \]
\end{definition}
\begin{definition}[Automorphism]
An automorphism of $G$ is an isomorphism from $G$ to $G$.

\textit{Note: $(Aut(G), \circ)$ forms a group.}
\end{definition}

Let $\mathfrak{S}(V)$ denote the symmetric group of all permutations (i.e. bijections) of the set $V$ (We also let $\mathfrak{S}_n$ denote the symmetric group of permutations of $[n]=\{ 1,2,...,n \}$). $Aut(G)$ is a subgroup of $\mathfrak{S}(V(G))$.

\begin{example}
$\mathrm{Aut(K_n)} = \mathfrak{S}_n$,
$\mathrm{Aut(C_n)} = D_n$(the dihedral group of order $2n$ of rotations and reflections of a regular $n$-gon).
\end{example}

\begin{theorem}[Lie Theory]
Miss this part...
\end{theorem}

\begin{example}
$n\geq 2$, $P_n$ has 2 automorphism (the identity map and the map that reflects the path about its center).
\end{example}

\begin{definition}[\(m\)-coloring]
\(m\)-coloring of a graph $G$ is a labeling of its vertices by a color $\varphi(V) \in [m]$.
An \(m\)-coloring is called \textbf{proper} if no two adjacent vertices are given the same color, i.e.,\(\varphi \text{ is proper }\) \iff \(\varphi\text{ is homomorphism form }G \text{ to }K_m\).
\end{definition}

\begin{definition}
A permutation matrix is a square $\{0,1\}$ matrix with exactly one $1$ in every row and column. For each permutation $\pi\in\mathfrak{S}_n$. Let $P(\pi)$ denote the $n\times n$ permutation matrix where $j^{\textrm{th}}$ column has a $1$ in row $\pi(j)$ for all $j\in[n]$.
\end{definition}
\begin{example}
Let $\pi=\begin{pmatrix}
1&2&3\\
2&3&1
\end{pmatrix}$, $\sigma=\begin{pmatrix}
1&2&3\\
2&1&3
\end{pmatrix}$. Then 
\[P(\pi)=\begin{pmatrix}
0&0&1\\
1&0&0\\
0&1&0
\end{pmatrix},\qquad 
P(\sigma)=\begin{pmatrix}
0&1&0\\
1&0&0\\
0&0&1
\end{pmatrix}.\]
Note that $P(\pi)P(\sigma)=P(\pi\sigma).$ Let $P_n$ be the set of all $n\times n$ permutation matrices. Then it's easy to verify that $P_n\leq \mathrm{GL}_n(\mathbb{R})$ and $P_n\cong \mathfrak{S}_n$.
\end{example}

\begin{definition}
Define the sign or signature $\mathrm{sgn}(\pi)$ of $\pi\in\mathfrak{S}_n$ by $\mathrm{sgn}(\pi)=\det(P(\pi))$.
\end{definition}

\begin{lemma}
Let $\pi\in\mathfrak{S}_n$, then
\begin{enumerate}[(i)]
\item If $\pi$ can be represented as a product of $m$ transpositions, then $\mathrm{sgn}(\pi)=(-1)^m$.
\item $\mathrm{sgn}(\pi) = (-1)^{\#\, i<j:\pi(j)>\pi(i)}$.
\end{enumerate}
\end{lemma}

\begin{proof}
\begin{enumerate}[(i)]
\item Using multiplicativity of the determinant, it suffices to prove that every transposition has sign $-1$. This is true since the permutation matrix of a transposition is
\[
\begin{pmatrix}
0&1&0&\cdots&0\\
1&0&0&\cdots&0\\
0&0&1&\cdots&0\\
\vdots &\vdots&\vdots&\ddots&\vdots\\
0&0&0&\cdots&1
\end{pmatrix}
,\]
up to a permutation of the basis vectors.
\item Check that 
\begin{enumerate}
\item $\mathrm{sgn}(\mathrm{id})=1$.
\item Since every transposition has sign -1, note that if there is inversion ($(i,j): i < j, \pi(j) > \pi(i)$) in a permutation, then there must exist an inversion between 2 consecutive indices (e.g. $\pi = 3124$ contains inversion (1,2) and (1,3)). Now it is enough to note that one could reverse such an inversion using transposition. So the total number of transpositions needed is the number of inversions.
\end{enumerate}
\end{enumerate}
\end{proof}

\begin{example}
$\mathrm{sgn}\begin{pmatrix}
1&2&3\\
2&3&1
\end{pmatrix}=1$
\end{example}

\begin{proposition}
Let $G$ and $H$ be graphs on the vertex set $[n]$ and $\pi\in\mathfrak{S}_n$. Then $\pi$ is an isomorphism from $G$ to $H$ if and only if 
 \[A(G)=P(\pi)^{-1}A(H)P(\pi).\]
\end{proposition}
\begin{proof}
Note that $(P(\pi)^{-1}A(H)P(\pi))_{i,j}=A(H)_{\pi(i),\pi(j)}.$ Therefore we have an equivalence chain,
\begin{align*}
&A(G)=P(\pi^{-1})A(H)P(\pi)\\
\Longleftrightarrow\quad & A(G)_{i,j}=A(H)_{\pi(i),\pi(j)}\\
\Longleftrightarrow\quad & (i\sim j\,\Longleftrightarrow\,\pi(i)\sim\pi(j))\\
\Longleftrightarrow\quad &\pi \textrm{ is an isomorphism}
\end{align*}
\end{proof}

\begin{corollary}
The automorphisms of a graph $G$ correspond to permutation matrices $P$ satisfying $A(G)=P^{-1}A(G)P$.
\end{corollary}

\begin{theorem}
Suppose that the eigenvalues of $G$ are all distinct. Then every automorphism of $G$ has order at most $2$.
\end{theorem}

\begin{proof}
Let $A=A(G)\in\mathbb{R}^{n\times n}$. Every automorphism of $G$ corresponds to a permutation matrix $P$ with $AP=PA$. We must show that $P^2=I_n$.

Let $\lambda_1,\ldots,\lambda_n\in\mathbb{R}$ be the eigenvalues of $A$ with eigenbasis $x^{(1)},\ldots,x^{(n)}\in\mathbb{R}^n$. Then for $i\in[n]$, we have 
\begin{align*}
&AP=PA\implies APx^{(i)}=PAx^{(i)}\implies A(Px^{(i)})=\lambda_i Px^{(i)}
\implies  Px^{(i)}=\mu_i x^{(i)}
\end{align*}
Since $P^{n!}=I_n$, $(\mu_i)^{n!}=1$. Thus $\mu_i=\pm 1$. We get $P^2x^{(i)}=\mu_i^2x^{(i)}=x^{(i)}$ for all $i\in[n]$. Since $\{x^{(i)}\}$ forms a basis of $\mathbb{R^n}$, $P^2$ is $I_n$.
\end{proof}
\begin{remark}
in fact, almost all graphs have a trivial automorphism group, i.e.
\[\lim_{n\to\infty}\frac{\#\textrm{graphs $G$ with $V(G)=[n]$ and $\left|\mathrm{Aut}(G)\right|=1$}}{\#\textrm{graphs $G$ with $V(G)=[n]$}}=1.\]
See section 2.3 of Godsil-Royle for a proof.
\end{remark}
\begin{remark}[Graph Isomorphic Problem]
\begin{flushleft}\end{flushleft}
\begin{itemize}
\item Babai \& Luks(1983)  $2^{O(\sqrt{n\log n})}$
\item Babai(2017) $2^{O((\log n)^c)}$
\end{itemize}
\end{remark}