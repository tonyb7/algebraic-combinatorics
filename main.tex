% This is the repository for Math 566 Algebraic Combinatorics Lecture Notes.

% The materials are voluntarily typed, and the professor are not likely to proofread them.

% If you are not familiar with LaTeX or don't want to type, you can click on the PDF on the black bar on the top of the page to download the PDF file.

% If you are willing to contribute by typing some lecture notes, click on the PROJECT icon on the black bar on the top of the page, and a side bar will occur on the left. Click on the corresponding lecture number and edit the file. You can refer to lecture01.tex as a sample.

% If you have some questions on the course materials, you can type the question in the question environment. It will be highlighted in red.

% If you have the answer to the questions, you can type the answer in the answer environment. It will be highlighted in green.

% If you are familiar with git commands, you can clone the repository and edit offline, and pull and push to synchronize.

\documentclass[12pt,oneside]{book}
\usepackage[letterpaper, margin=1in, headheight=105pt]{geometry}
\usepackage{subcaption}
\usepackage{enumerate}
\usepackage{mathtools}
\usepackage{listings}
\usepackage{calligra}
\usepackage{fancyhdr}
\usepackage{titlesec}
\usepackage{graphicx}
\usepackage{amsmath}
\usepackage{amssymb}
\usepackage{caption}
\usepackage{titling}
\usepackage{comment}
\usepackage{xparse}
\usepackage{amsthm}
\usepackage[dvipsnames]{xcolor}
\usepackage[colorlinks=true,
			linkcolor=WildStrawberry,
			anchorcolor=red,
			citecolor=green,
			urlcolor=RoyalBlue,
			filecolor=brown,
			menucolor=pink]{hyperref}
\usepackage[linesnumbered,
			boxed,
			algochapter,
			algo2e,
			lined]{algorithm2e}
\usepackage[numbered,
			framed]{matlab-prettifier}
\usepackage[backend=biber,backref=true]{biblatex}
\usepackage[titletoc]{appendix}
\addbibresource{reference.bib}

%% tikz setting for graph drawing
\usepackage{tikz}
\usetikzlibrary{positioning}
\tikzset{plain/.style={circle,fill=blue!20,draw,minimum size=0.5cm,inner sep=0pt},
}

\makeatletter
\@addtoreset{chapter}{part}
\makeatother

\title{\textbf{Math 566 Algebraic Combinatorics}}
\author{\calligra{Steven Karp}}
\date{\today}


\newcommand{\Z}{\ensuremath{\mathbb{Z}}}
\newcommand{\R}{\ensuremath{\mathbb{R}}}
\newcommand{\N}{\ensuremath{\mathbb{N}}}
\newcommand{\Q}{\ensuremath{\mathbb{Q}}}
\newcommand{\C}{\ensuremath{\mathbb{C}}}
\DeclareMathOperator*{\E}{\ensuremath{\mathbb{E}}}
\renewcommand{\iff}{\ifmmode \iff \else \text{if and only if} \fi}


\newtheorem{theorem}{Theorem}[chapter]
\newtheorem{lemma}{Lemma}[chapter]
\newtheorem{proposition}{Proposition}[chapter]
\newtheorem{corollary}{Corollary}[chapter]
\theoremstyle{definition}
\newtheorem{definition}{Definition}[chapter]
\newtheorem{example}{Example}[chapter]
\theoremstyle{remark}
\newtheorem*{remark}{Remark}

\newtheoremstyle{ques}{}{}{\color{red}}{}{\bfseries\color{red}}{:}{ }{}
\theoremstyle{ques}
\newtheorem*{question}{Question}

\newtheoremstyle{ans}{}{}{\color{green}}{}{\bfseries\color{green}}{:}{ }{}
\theoremstyle{ans}
\newtheorem*{answer}{Answer}

\graphicspath{{./graphics/}}

\setlength{\algomargin}{2em}
\lstset{xleftmargin=.1\textwidth, xrightmargin=.1\textwidth}

\pagestyle{fancy}


\begin{document}

\chapter{Algebraic Graph Theory}

\begin{itemize}
\item Linear algebra
\item Group theory (Cayley graphs, Dynkin diagrams)
\end{itemize}

\section{Eigenvalues}

\begin{definition}[Adjacency matrix]
Let \(G=(V,E)\) be a finite graph.
The \emph{adjacency matrix} \( A(G)=\{0,1\}^{V\times V} \) is defined by
\[ A(G)_{v,w}=\begin{cases} 1 & \text{if } v\sim w \\ 0 & \text{otherwise} \end{cases} \]
\end{definition}

Recall the characteristic polynomial of a square matrix \(M\) over \C
\[ \phi_M(t)\coloneqq \det(tI-M). \]


\begin{definition}[Characteristic polynomial]
Let \(G=(V,E)\) be a finite graph.
The \emph{Characteristic polynomial} is defined by
\[ \phi_G(t)\coloneqq \phi_{A(G)}(t), \]
and call the zeros (with multiplicitier) the eigenvalues of \(G\).
\end{definition}

\begin{definition}[Spectrum]
The \emph{spectrum} of \(G\) is the multiset of its eigenvalues.
\end{definition}

\begin{example}[spectrum]
\begin{align*}
A(G)&=\begin{bmatrix} 0 & 1 \\ 1 & 0 \end{bmatrix}\\
\phi_G(t)&=\det\begin{pmatrix} t & -1 \\ -1 & t \end{pmatrix}=-t^2-1\\
\text{spectrum}&: \{1,-1\}
\end{align*}
\end{example}

\begin{remark}
Eigenvalues provide information about the connectivity of the graph.
\end{remark}


\begin{theorem}
Let \(A\in\R^{n\times n}\) be symmetric, then \(A\) \emph{orthogonally diagonalizable} over \R, \text{i.e.},
there exists an orthonormal basis
\[ v^{(1)}, v^{(2)},\cdots,v^{(n)}\in\R^n \]
of eigenvectors of \(A\) corresponding to real eigenvalues \(\lambda_1,\lambda_2,\cdots,\lambda_n\in\R\), we have
\[A=\sum_{i=1}^{n}\lambda_i v^{(i)}v^{(i)T} \]
\end{theorem}

Therefore the eigenvalues of any graph \(G\) are all real and we'll denote then
\[ \lambda_1(G)\geq\lambda_2(G)\geq\cdots\geq\lambda_n(G), \]
where \(n=|V(G)|\).

\begin{theorem}[Perron-Forbenius]
If a matrix \(A\in\R^{n\times n}\) has nonnegative entries, then the spectral radius of \(A\) (\textit{i.e.}, the maximum magnitude over all complex eigenvalues of A) is an eigenvalue of \(A\), corresponding to an eigenvector in \(\R_{\ge0}^n\).
\end{theorem}
Therefore for any graph \(G\), \(\lambda_1(G)\) is the spectral radius and corresponds to an eigenvector with nonnegative entries.
Perron-Forbenius also implies if \(G\) is connected, then \(\lambda_1(G)\) has multiplicity 1.


\begin{definition}[disjoint union]
If \(G=(V,E)\), \(G'=(V',E')\) are graphs, their \emph{disjoint union} is the graph
\[ G\sqcup G'=(V\sqcup V', E\sqcup E') \]
and
\[ A(G\sqcup G)=\begin{bmatrix} A(G) & 0 \\ 0 & A(G') \end{bmatrix} \]
\end{definition}
\begin{remark}
Spectrum doesn't detect if the graph is connected.
\end{remark}
\begin{example}
\begin{align*}
G&: V=\{1,2\}, E=\{\{1,2\}\}\\
\text{spectrum}&: \{1,-1\}\\
G\sqcup G&: V=\{1,2,3,4\}, E=\{\{1,2\},\{3,4\}\}\\
\text{spectrum}&: \{1,1,-1,-1\}
\end{align*}
\end{example}

\begin{remark}
Note that the spectrum of \(G\sqcup G'\) is the multiset union of the spectra of \(G\) and \(G'\).
\end{remark}


\section{Regular Graphs}
\begin{definition}
A graph \(G=(V,E)\) is called \emph{\(k\)-regular} if every vertex has degree \(k\in\N\).
\end{definition}
\begin{remark}
\(G\) is regular \iff \(\bar e\) is an eigenvector, where \(\bar e=\begin{bmatrix} 1 & 1 & \cdots & 1 \end{bmatrix}^T\in\R^n \).
\[ A(G)\cdot\bar e=\deg_G(V) \]
\end{remark}

\begin{proposition}
Let \(G=(V,E)\), then
\[ \E_{v\in V(G)} \deg_G(v) \stackrel{(i)}{\leq} \lambda_1(G) \stackrel{(ii)}{\leq} \max_{v\in V(G)}\deg_G(V), \]
where
\[ \E_{v\in V(G)} \deg_G(v)=\frac{1}{n}\sum_{v\in V(G)}\deg_G(v). \]
We have equality in (i) \iff \(G\) is regular.
If \(G\) is connected, equality holds in (ii) \iff \(G\) is regular.
\end{proposition}
\begin{proof}
\begin{enumerate}[(i)]
\item For any symmetric \(A\in\R^{n\times n}\),
\[ \lambda_{\max}(A)\geq x^T Ax, \forall x\in\R^n \text{ with } ||x||=1 \]
with equality \iff \(x\) is an eigenvector of \(A\) with eigenvalue \(\lambda_{\max}(A)\).
\begin{align*}
A&=\sum_j \lambda_j x^{(j)}x^{(j)T}\\
x&=\sum_j c_j x^{(j)}\\
\sum_j c_j^2&=1\\
x^T Ax&=\sum_j \lambda_j c_j^2
\end{align*}
Take \(A=A(G), x=\frac{1}{\sqrt{n}}\bar e\), then
\[ x^T Ax=\frac{1}{n} \underbrace{\bar e^T A\bar e}_{\text{sum of entries}}=\text{average degree} \]
\item Let's assume \(V(G)=[n]=\{1,2,\cdots,n\}\).
Recall that by Perron-Forbenius, \(\lambda_1(G)\) corresponds to an eigenvector \(x\in\R_{\ge0}^n\).
Then for any \(i\in[n]\),
\[ \lambda_1(G) x_i=\Big(A(G)x\Big)_i=\sum_{j=1}^{n} A(G)_{i,j}x_j=\sum_{j\sim i}x_j. \]
Now take \(i\in[n]\) so that \(x_i\) is maximum.
Then
\[ \lambda_1(G)x_i=\sum_{j\sim i}x_j\leq \sum_{j\sim i}x_i=\deg_G(v_i) x_i\leq \left( \max_{v\in V(G)}\deg_G(v) \right) x_i \tag{*} \]
So
\[ \lambda_1(G)\leq\max_{v\in V(G)}\deg_G(v). \]
Suppose equality holds everywhere in (ii) and \(G\) is connected, then equality holds everywhere in (*).
So \(x_j=x_i\) for all \(j\sim i\).
Applying the same argument to \(j\sim i\), then \(x_h=x_j=x_i\) for any \(h\sim j\), \texttt{etc}.
Since \(G\) is connected, \(x\) is a multiple of \(\bar e\), \textit{i.e.}, \(\bar e\) is an eigenvector of \(G\).
\end{enumerate}
\end{proof}
\begin{corollary}
The complete graph \(K_n\) is the only graph on \(n\) vertices with eigenvalue \(n-1\).
In particular, \(K_n\) is uniquely determined by its spectrum.
\end{corollary}
\begin{remark}
Not all graphs can be recovered by their spectrum, \textit{i.e.}, the spectrum is not a faithful (in that graph can be uniquely determined) graph invariant (in that spectrum depends on graph only up to isomorphism).
\end{remark}
\begin{example}
	Spectrum does not tell connectivity. For example, the following two graphs have the same spectrum $(-2,2,0,0,0)$.
	\begin{figure}[tbp]
	\centering
	\begin{tikzpicture}
	%% disconnected graph
	\node[plain](1) {};
	\node[plain](2) [right = 1cm of 1]  {};
	\node[plain](3) [above = 1cm of 1] {};
	\node[plain](4) [left = 1cm of 1] {};
	\node[plain](5) [below = 1cm of 1] {};
	\path[draw,thick]
	(2) edge node {} (3)
	(3) edge node {} (4)
	(4) edge node {} (5)
	(5) edge node {} (2);
	%% connected graph
	\begin{scope}[xshift=6cm]
	\node[plain](1) {};
	\node[plain](2) [right = 1cm of 1]  {};
	\node[plain](3) [above = 1cm of 1] {};
	\node[plain](4) [left = 1cm of 1] {};
	\node[plain](5) [below = 1cm of 1] {};
	\path[draw,thick]
	(1) edge node {} (3)
	(1) edge node {} (4)
	(1) edge node {} (5)
	(1) edge node {} (2);
	\end{scope}
	\end{tikzpicture}
    \label{fig:spectrum_connectivity_example}
    \caption{Two graphs with the same spectrum}
	\end{figure}
\end{example}
\section{Bipartite Graphs}
Recall graph is bipartite with biparts \(x,y\) \iff
\[ A(G)=\begin{bmatrix}0 & B \\ B^T & 0\end{bmatrix} \]

\begin{lemma}
Let \(B\) be an \(n\times n\) matrix such that \(B^TB\) has nonzero eigenvalues \(\lambda_1,\cdots,\lambda_r\) (with multiplicities), then the nonzero eigenvalues of 
\[ A=\begin{bmatrix}0 & B \\ B^T & 0\end{bmatrix}\in\R^{(m+n)\times(m+n)} \]
are precisely \( \sqrt{\lambda_1},-\sqrt{\lambda_1},\cdots,\sqrt{\lambda_r},-\sqrt{\lambda_r} \) (with multiplicities).
\end{lemma}
\begin{remark}
The eigenvalues of \(B^TB\) are real (from symmetry) and nonnegative since if \(B^T Bx=\lambda x\) (we assume \(||x||=1\))
\[ \lambda =\lambda x^T x= x^T (\lambda x)= x^T(B^T Bx) =(Bx)^T(Bx)=\langle Bx,Bx\rangle\geq0. \]
\end{remark}
\begin{proof}
Note that
\[ \begin{bmatrix}t I_m & -B \\ -B^T & t I_n\end{bmatrix}\begin{bmatrix} I_m & B \\ 0 & t I_n\end{bmatrix}=\begin{bmatrix}t I_m & 0 \\ -B^T & t^2 I_n-B^TB \end{bmatrix}. \]
Take determinant gives
\[ \phi_A(t) t^n=t^m\phi_{B^TB}(t^2) \]
\end{proof}

\begin{example}
Let \(K_{m,n}\) be the complete bipartite graph with biparts of size \(m\) and \(n\) and all possible edges between them, then
\[ A(K_{m,n})=\begin{bmatrix}0 & B \\ B^T & 0\end{bmatrix} \]
where \(B\) is an \(m\times n\) all-one matrix. We have
\[ \underbrace{B^T B}_{n\times n} = m J_n \]
where \(J_n\) is the \(n\times n\) all-one matrix.
Recall that \(J_n\) has exactly one nonzero eigenvalues namely \(n\), so the eigenvalues of \(K_{m,n}\) are \(\sqrt{mn},-\sqrt{mn}\) and 0 (multiplicity \(m+n-2\)).
\end{example}

\begin{example}
Let \(C_{2n}\) be a bipartite graph with adjacency matrix 
\[ A(C_{2n})=\begin{bmatrix}0 & B \\ B^T & 0\end{bmatrix} \]
where \(B\in\R^{n\times n}\) \textit{e.g.}
\[B = \begin{bmatrix}1&0&0&0&1 \\ 1&1&0&0&0 \\ 0&1&1&0&0 \\ 0&0&1&1&0 \\ 0&0&0&1&1 \end{bmatrix}\in\R^{5\times 5} \]
We can verify that
\[ B^T B=2I_n+\underbrace{A(C_n)}_{\begin{bmatrix}0&1&0&0&1 \\ 1&0&1&0&0 \\ 0&1&0&1&0 \\ 0&0&1&0&1 \\ 1&0&0&1&0 \end{bmatrix}} \]
Therefore if \(\lambda_1,\cdots,\lambda_n\) are the eigenvalues of \(C_n\), then the eigenvalues of \(C_{2n}\) are
\[ \pm\sqrt{2+\lambda_k}, \quad k=1,\cdots,n \]
This agrees with our early calculations
\[ \lambda_k=2\cos\left(\frac{2\pi k}{n} \right) \]
with identity
\[ 1+\cos(2\theta)=2\cos^2\theta \]
\end{example}


\begin{corollary}
If \(G\) is bipartite graph on \(n\) vertices then \(\lambda_n(G)=-\lambda_1(G)\).
\end{corollary}
\begin{remark}
A converse holds for connected graphs.
\end{remark}
\begin{proposition}
If \(G\) is connected on \(n\) vertices and \(\lambda_n(G)=-\lambda_1(G)\), then \(G\) is bipartite.
\end{proposition}
\begin{remark}
It's another consequence of the Perron-Frobenius theorem. We don't prove this.
\end{remark}



\section{Cartesian Products}

\begin{definition}[Cartesian product]
Let \(G\) and \(H\) be graphs, the \emph{Cartesian product} \(G\square H\) has vertex set \(V(G)\times V(H)\) with edges of the forms
\begin{itemize}
\item \((v,w)\sim(v',w)\) where \(v\sim v'\) in \(G\) and \(w\in V(H)\)
\item \((v,w)\sim(v,w')\) where \(w\sim w'\) in \(H\) and \(v\in V(G)\)
\end{itemize}
\end{definition}

\begin{example}
\(P_m\square P_n\) is the \(m\times n\) rectangular lattice.
\(P_2\square P_2\square P_2\) is the 1-skeleton of cude.
In general, \(\underbrace{P_2\square\cdots\square P_2}_{n\text{ copies}}\) is the 1-skeleton of the \(n\)-dimensional hypercube \([0,1]^n\subset\R^{n}\)
\end{example}

\begin{definition}[Tensor product]
Let \(\R^m\otimes\R^n\) denote the tensor product of \(\R^m\) and \(\R^n\) which we will identify with the vector space of matrices \(\R^{m\times n}\).
Let \(e^{(i)}\) denote the unit vector (\(i^\text{th}\) entry is one and zeros elsewhere) and define the standard basis of \(\R^{m\times n}\):
\[ \left\{ e^{(i)}e^{(j)T}=\kbordermatrix{ &  &  & j &  & \\  & 0 & 0 & 0 & 0 & 0 \\  & 0 & 0 & 0 & 0 & 0 \\ i & 0 & 0 & 1 & 0 & 0 \\ & 0 & 0 & 0 & 0 & 0 \\ & 0 & 0 & 0 & 0 & 0}, 1\leq i\leq m,1\leq j\leq n \right\} \]
For \(A\in\R^{m\times m},B\in\R^{n\times n}\), let \(A\otimes B\) denote the endomorphism of \(\R^{m\times n}\) given by
\[ (A\otimes B)(M)\coloneqq AMB^T \quad \text{for } M\in\R^{m\times n}. \]
\end{definition}

\begin{lemma}
Let two matrices \(A\in\R^{m\times m},B\in\R^{n\times n}\), then
\[ (A\otimes B)_{(i,j),(k,l)}=A_{i,k}B_{j,l}, \quad (1\leq i,k\leq m,1\leq j,l\leq n) \]
\end{lemma}
\begin{proof}
Rercall that the entries of a matrix \(M\in\R^{d\times d}\) are characterized by
\[ Me^{(j)}=\sum_{i=1}^{d}M_{i,j}e^{(i)} \quad \text{for } 1\leq j\leq d. \]
\begin{align*}
(A\otimes B)\left(e^{(k)}e^{(l)T}\right)&=A\left(e^{(k)}e^{(l)T}\right)B^T\\
&=\left(Ae^{(k)}\right)\left(Be^{(l)}\right)^T\\
&=\left(\sum_{i=1}^m A_{i,k}e^{(i)}\right)\left(\sum_{j=1}^n B_{j,l}e^{(j)}\right)^T\\
&=\sum_{i=1}^m \sum_{j=1}^n A_{i,k}B_{j,l} e^{(i)}e^{(j)T}
\end{align*}
\end{proof}

\begin{corollary}
Let \(G\) and \(H\) be graphs with basis of eigenvectors \(x^{(1)},\cdots,x^{(m)}\in\R^m\) and \(y^{(1)},\cdots,y^{(n)}\in\R^n\) cooresponding to eigenvalues \(\lambda_1,\cdots,\lambda_m\) and \(\mu_1,\cdots,\mu_n\), then
\[ A(G\square H)=A(G)\otimes I_n + I_m\otimes A(H) \]
and \(G\square H\) has eigenvectors \(x^{(i)}y^{(j)T}\) corresponding to eigenvalues \(\lambda_i+\mu_j\) for \(1\leq m,1\leq j\leq n\).
\end{corollary}
\begin{remark}
Note that \(\left\{x^{(i)}y^{(j)T}\right\}\) forms a basis of \(\R^{m\times n}\) since they are the unit matrices (matrices with a 1 and 0's elsewhere) under the basis \(x^{(1)},\cdots,x^{(m)}\) of \(\R^m\) and the basis dual to \(y^{(1)},\cdots,y^{(n)}\in\R^n\).
(If \(y^{(1)},\cdots,y^{(n)}\in\R^n\) are orthonormal, it's self-dual).
\end{remark}

\end{document}